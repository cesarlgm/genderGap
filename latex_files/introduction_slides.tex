\section{Introduction}


\begin{frame}{Introduction}
	\bitem
		\item There is a wage premium from living in a large labor market [citations here].
		\item But little attention has been paid as to how this premium varies by gender.

	In this paper I document three facts that link the gender wage-gap with the urban wage premium:
	\benu
	\item Today, women \textbf{\alert{gain relatively}} more from being in a denser labor market $\implies$ denser labor markets have \alert{\textbf{a narrower gender-wage gap}}.
	\item Women relative gain from being in a denser labor market has increased since 1970 $\implies$ the gender gap-density elasticity went from \textbf{\alert{0.04}} to \textbf{\alert{-0.01}}. 
	\item sth about possible channels here
	\eenu
	\eitem
\end{frame}


%\begin{frame}{Summary} 
%	In the next slides I document three main facts about the \alert{\textbf{gender gap}} in the US for the period of 1970 and 2020:
%	\benu
%		\item There is a large dispersion in the \textbf{\alert{level}} of the gender wage gap across labor markets in the US. The dispersion persists despite the general decrease in the level of the gap since 1970.
%		\item There are differences in the \textbf{\alert{change}} of the gender wage gap. The largest reductions happened in densest labor markets.
%		\item The relationship between the \textbf{\alert{level}} of gender wage gap and population density has inverted over the period. Today, the densest labor markets have a lower gender wage gap.
%	\eenu
%\end{frame}

\begin{frame}{Literature}
	\bitem
		\item Gender gap literature
		\item Decline in the urban wage premium
		\item Urban literature.
	\eitem 
	
\end{frame}