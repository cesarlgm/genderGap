\section{Descriptive analysis}
\subsection{Basic facts and trends}
\bitem
\item Most CZ are relative small. 40\% of CZ account for 85\% of the of the total population in almost all years.
\item CZ at the very top have lost population share. "Mid"-tier cities are the ones with the fastest population growth.
\item Overall, the US has experienced manufacturing decline. But decline has been much faster in the top 30\% CZ (see \href{https://www.dropbox.com/s/cp9v7uk87lyij4r/bar_deciles_manufacturing.png?dl=0}{here}).
\item 

\eitem

\subsection{The gender gap}
\bitem
\item Wage gap is decreasing everywhere. But decline is faster in densest CZ.
\item There is a clear inversion of the gender-gap density gradient. 
\item This inversion is also present if I focus attention to the top 248 CZ. 
[see this \href{https://www.dropbox.com/s/yytq0djtbb8lb5i/bar_graph_deviation_from_mean.png?dl=0}{graph}].

\item
\eitem
\subsection{What is happening to the urban wage premium of each gender?}
There are two distinct periods:
\bitem
	\item \textbf{1970-1990:} this is a women's progress period. Density premium is increasing for both genders, but the increase faster for men.
	\item \textbf{1990-2020:} this looks a clear relative male decline story.
\eitem

What is happening to the absolute wages of these people?

\bitem
	\item \textbf{Women:} faster increase in denser places from 70s - 90s. Increase is relatively flat afterwards. See \href{https://www.dropbox.com/s/cwuy5a3rduk6ph9/d_wage_deciles_women.png?dl=0}{here}.
	\item \textbf{Men:} there is a clear decline in denser CZ after 2000. See \href{https://www.dropbox.com/s/19e7o1raaanezug/d_wage_deciles_men.png?dl=0}{here}.
\eitem 


\subsection{LFP}
\bitem
	\item Employment to population ratio has always had an U pattern for men. The U pattern has exacerbated over the years. Decline in middle places has been faster. This is in line with the Ely Lecture (see \href{https://www.dropbox.com/s/w0m6oerp6mt7edg/bar_graph_lfp_male.png?dl=0}{here})
	\item For women, densest places offered more employment at the start. Over time, they have become more like men. U pattern seems to be intriguing  (see \href{https://www.dropbox.com/s/mhlqjp6k73i9ub4/bar_graph_lfp_women.png?dl=0}{here})
	\item What about the absolute values:
	\bitem
		\item For men, the story if one of decline in LFP. With faster decline in the middle of the distribution. See \href{https://www.dropbox.com/s/rdhcj6im9xt1ad3/bar_graph_d_lfp_male.png?dl=0}{here}
		\item For women, the story is one of faster progress at the tails of the distribution. \href{https://www.dropbox.com/s/sl9vlat28ongdlb/bar_graph_d_lfp_female.png?dl=0}{here}
	\eitem
\eitem 


